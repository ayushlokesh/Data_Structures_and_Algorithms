\documentclass[a4]{article}
\usepackage{gnuplottex}
\usepackage{csvsimple}
\usepackage{subcaption}
\title{COMP26120 Lab 2 Report}
\author{Your Name Here}
\begin{document}
\maketitle

\section{Experiment 1 -- Sorting Performance}

\subsection{Theoretical Best Case}

\paragraph{Hypothesis} The behaviour for insertion sort on sorted input is linear. 

%% Write a brief paragraph here to justify this hypothesis with reference to the theoretical complexity.

\paragraph{Experimental Design} 

%% Your experimental design should include the following

%% Your intended independent variable -- this is the thing you are varying

%% Your intended dependent variable -- this is the thing you are measuring whose value your hypothesis says depends upon the value of the independent variable.

%% Anything that you could vary but you are not -- for instance to avoid confusing your results with other variables that might also influence performance.

%% What data you generated and how many examples for each value of your independent variable.

%% What program you ran on what inputs.  How many times you ran it on each input.

%% What you measured and how.

%% How you propose to determine $f(x)$


\paragraph{Results} 

%% Ideally you should include a graph of your results showing the best fit line for f.

%% A statement of what the results have determined $f(x)$ to be

%% You should state the values you have determined for $c_1$ and $c_2$.

%% You should state clearly whether your hypothesis was confirmed or refuted.

\subsection{Theoretical Worst Case}

\paragraph{Hypothesis} The behaviour for insertion sort on reverse sorted input is quadratic.  

%% Write a brief paragraph here to justify this hypothesis with reference to the theoretical complexity.

\paragraph{Experimental Design} 

%% Your experimental design should include the following

%% Your intended independent variable -- this is the thing you are varying

%% Your intended dependent variable -- this is the thing you are measuring whose value your hypothesis says depends upon the value of the independent variable.

%% Anything that you could vary but you are not -- for instance to avoid confusing your results with other variables that might also influence performance.

%% What data you generated and how many examples for each value of your independent variable.

%% What program you ran on what inputs.  How many times you ran it on each input.

%% What you measured and how.

%% How you propose to determine $f(x)$

\paragraph{Results} 

%% Ideally you should include a graph of your results showing the best fit line for f.

%% A statement of what the results have determineed $f(x)$ to be

%% You should state the values you have determined for $c_1$ and $c_2$.

%% You should state clearly whether your hypothesis was confirmed or refuted.

\subsection{Average Case}

\paragraph{Hypothesis} The behaviour for insertion sort on random input is somewhere between the performance on sorted and reverse sorted input.  

%% Write a brief paragraph here to justify this hypothesis with reference to the theoretical complexity.

\paragraph{Experimental Design} 

%% Your experimental design should include the following

%% Your intended independent variable -- this is the thing you are varying

%% Your intended dependent variable -- this is the thing you are measuring whose value your hypothesis says depends upon the value of the independent variable.

%% Anything that you could vary but you are not -- for instance to avoid confusing your results with other variables that might also influence performance.

%% What data you generated and how many examples for each value of your independent variable.

%% What program you ran on what inputs.  How many times you ran it on each input.

%% What you measured and how.

%% How you propose to determine $f(x)$.  In this case you will have to describe how you will decide whether a linear or quadratic line was a better fit.

\paragraph{Results} 

%% Ideally you should include a graph of your results showing the best fit line for f.  Ideally this graph will compare with the performance for the best and worst cases.

%% A statement of what the results have determineed $f(x)$ to be

%% You should state the values you have determined for $c_1$ and $c_2$.

%% You should state clearly whether your hypothesis was confirmed or refuted.

\subsection{Discussion}

%% Write a short paragraph discussing your results more generally.  Did sorted data turn out to be best case?  Did reverse sorted data turn out to be worst case?  Was the performance for average case more like best or worst case?  If things didn't behave as you expected can you speculate on why?  What further experiment(s) might you run to help determine why?

\section{Experiment 2}

\paragraph{Hypothesis} %% You want a hypothesis here probably expressed in terms of k (the dictionary size) and n (the query size) that suggests when you think it will become worth sorting.  You probably want to express this as a range of values and you will use your experiment to find exactly where in that range the point occurs.

%% You should then back the hypothesis up with some fuller discussion on how you computed the range where the crossover point was likely to occur based on your knowledge of the complexity of searching and sorting.

\paragraph{Experimental Design} 

%% Your experimental design should include the following

%% Your intended independent variable -- this is the thing you are varying

%% Your intended dependent variable -- this is the thing you are measuring whose value your hypothesis says depends upon the value of the independent variable.

%% Anything that you could vary but you are not -- for instance to avoid confusing your results with other variables that might also influence performance.

%% What data you generated and how many examples for each value of your independent variable.

%% What program you ran on what inputs.  How many times you ran it on each input.

%% What you measured and how.

%% How you will determine the best fit $f(x)$ for each case.  

%% How you will determined where sorting became "worth it"


\paragraph{Results} 

%% Ideally you should include a graph of your results.  You may need several graphs for this experiment showing several different set ups.

%% You should state clearly whether your hypothesis was confirmed or refuted.

%% You should state the value(s) you have determined for when sorting becomes worth it.


\subsection{Discussion} %% Optional section

%% Depending on your results you may need some additional reflection -- for instance to hypothesise why things didn't behave as expected.  Or reflect on why you got different points where sorting became worth it for different set ups.

\section{Data Statement}

%% It is good practice to inform readers where they can find your data and code in order to check your results and re-run experiments for themselves.  This should appear in a Data statement.

%% We've asked you not to include large input files in your gitlab, but you should include any scripts you created - for instance to generate input files or run experiments, and the "raw" data - e.g., a table (e.g., as a .csv file or spreadsheet) of independent variable values matched with dependent variable values.  You may also include this as an appendix in the report.

%% The data statement should clearly state where all the input data (if included), scripts and output data can be found.


\appendix
%% As appendices you may want to include tables of raw data

%% scripts used (not including those we supplied) to generate dictionaries, queries and results.

\end{document}