\documentclass[a4]{article}
\usepackage{gnuplottex}
\usepackage{csvsimple}
\usepackage{subcaption}
\title{COMP26120 Lab 3 Report}
\author{Your Name Here}
\begin{document}
\maketitle

\section{Experiment 1}


\paragraph{Hypothesis} The complexity of inserting a single value into my implementation of a hash set is ...

%% Complete this sentence then write a brief paragraph here to justify this hypothesis with reference to the theoretical complexity.

\paragraph{Experimental Design} 

%% Your experimental design should include the following

%% Your intended independent variable -- this is the thing you are varying

%% Your intended dependent variable -- this is the thing you are measuring whose value your hypothesis says depends upon the value of the independent variable.

%% Anything that you could vary but you are not -- for instance to avoid confusing your results with other variables that might also influence performance.

%% What data you generated and how many examples for each value of your independent variable.

%% What program you ran on what inputs.  How many times you ran it on each input.

%% What you measured and how.


\paragraph{Results} 

%% Ideally you should include a graph of your results showing a best fit line.

% If you do any processing on results, such as generating best fit lines, computing averages, etc., then these should be described. 

%Raw data should be presented in an appendix or included with your code submission. 

% You should state clearly whether your hypothesis was confirmed or refuted (or it is equivocal or difficult to tell). If your hypothesis was refuted you should discuss why that might be (e.g., incorrect implementation of insert, some problem with the experimental design) and sketch how you might investigate further. 

\section{Experiment 2}

\paragraph{Hypothesis}  The behaviour of insert for binary tree with unsorted input is...

%% Complete this sentence then write a brief paragraph here to justify this hypothesis with reference to the theoretical complexity.

\paragraph{Experimental Design} 

%% Your experimental design should include the following

%% Your intended independent variable -- this is the thing you are varying

%% Your intended dependent variable -- this is the thing you are measuring whose value your hypothesis says depends upon the value of the independent variable.

%% Anything that you could vary but you are not -- for instance to avoid confusing your results with other variables that might also influence performance.

%% What data you generated and how many examples for each value of your independent variable.

%% What program you ran on what inputs.  How many times you ran it on each input.

%% What you measured and how.


\paragraph{Results} 

%% Ideally you should include a graph of your results showing a best fit line.

% If you do any processing on results, such as generating best fit lines, computing averages, etc., then these should be described. 

%Raw data should be presented in an appendix or included with your code submission. 

% You should state clearly whether your hypothesis was confirmed or refuted (or it is equivocal or difficult to tell). If your hypothesis was refuted you should discuss why that might be (e.g., incorrect implementation of insert, some problem with the experimental design) and sketch how you might investigate further. 


\section{Conclusion} 

% Use your results to address the initial question: when should you use your hash set implementation and when should you use your binary tree implementation? You do not need to validate this experimentally since for some implementations this is likely to take a lot of time, even with a good implementation on a fast machine. Note that, theoretically, there is a crossover point where one becomes better than the other though, depending upon your results, this may be difficult to calculate since it may be very small or very large - in this case you should at least discuss how you might try to re-design or extend your experiments and how you would then calculate the crossover point

.\section{Data Statement}

%% It is good practice to inform readers where they can find your data and code in order to check your results and re-run experiments for themselves.  This should appear in a Data statement.

%% We've asked you not to include large input files in your gitlab, but you should include any scripts you created - for instance to generate input files or run experiments, and the "raw" data - e.g., a table (e.g., as a .csv file or spreadsheet) of independent variable values matched with dependent variable values.  You may also include this as an appendix in the report.

%% The data statement should clearly state where all the input data (if included), scripts and output data can be found.


\appendix
%% As appendices you may want to include tables of raw data

%% scripts used (not including those we supplied) to generate dictionaries, queries and results.

\end{document}